%	$Id: format-p.tex,v 5.0 2001/06/19 08:22:47 robert Exp $
%
%	Documentation for format-p
%
%%%%%%%%%%%%%%%%%%%%%%%%%%%%%%%%%%%%%%%%%%%%%%%%%%%%%%%%%%%%%%%%%%%%%%%%%%%%%%%%
%	$Log: format-p.tex,v $
%	Revision 5.0  2001/06/19 08:22:47  robert
%	LS10-5.0 New Release as of 19 JUNE 2001
%	
%	Revision 4.0  2001/03/09 02:43:47  scott
%	LS10-4.0 New Release as at 10th March 2001
%	
%	Revision 3.1  2000/11/10 05:01:11  scott
%	Updated to allow for local printing.
%	
%	Revision 3.0  2000/10/10 12:24:13  gerry
%	Revision No. 3 Start
%	<after Rel-10102000>
%	
%	Revision 2.0  2000/07/15 09:15:12  gerry
%	Forced Revision No Start 2.0 Rel-15072000
%	
%	Revision 1.1.1.1  1999/07/15 00:19:15  jonc
%	Adopted from Pinnacle V10
%	
%	Revision 1.9  1999/04/22 22:25:02  jonc
%	Begin section form application interface. Still needs more work.
%	
%	Revision 1.8  1998/08/11 02:48:17  jonc
%	Minor editing changes.
%	
%	Revision 1.7  1998/07/22 22:31:04  jonc
%	Minor formatting change.
%
%	Revision 1.6  1998/07/22 22:16:30  jonc
%	Moved pitch setting to layout-file.
%
%	Revision 1.5  1998/07/20 03:55:03  jonc
%	Took `pformat' out of the process chain.
%
%	Revision 1.4  1998/07/09 22:51:51  jonc
%	Added support for %reset directive.
%
%	Revision 1.3  1998/07/01 03:55:05  jonc
%	Rehashed versioning style and document date.
%
%	Revision 1.2  1998/06/30 00:56:08  jonc
%	Added %directive feature.
%
%	Revision 1.1  1998/06/29 21:20:35  jonc
%	First cut.
%

%
% Preamble
\documentclass[a4paper,twoside]{article}

\usepackage{alltt}

\title
{
	Format-P \\
	\small{Document $ $Revision: 5.0 $ $}
}

\author{Jonathan Chen}

\date
{
	\small{$ $Date: 2001/06/19 08:22:47 $ $ GMT}
}

%
% Actual document
\begin{document}

\maketitle							% generate a title page

\section{Overview}

\texttt{format-p} is my latest attempt at a generic text formatter
that is intended to replace \texttt{pformat}. \texttt{format-p} utilises
the same configuration files as \texttt{pformat} to determine the printer
output queue and
generate escape codes for pitch changes; so any changes to the system
for \texttt{pformat} will also affect \texttt{format-p} as well.

\texttt{format-p} uses a layout
file to determine how to present the output. With information provided
from an input stream (generally \emph{standard-input}), it sets up and
substitutes \emph{value-registers} defined within the layout file.

\texttt{format-p} divides a page up into 3 sections: The Page-Header, the Body
and the Page-Trailer. As it prints out each Body section (per detail
line), it checks to see whether additional page structures such as the
Page Header and Page Trailer need to be generated.

%%%%%%%%%%%%%%%%%%%%%%%%%%%%%%%%%%%%%%%%%%%%%%%%%%%%%%%%%%%%%%%%%%%%%%
\section{The Layout File}

The layout file consists of named blocks of information enclosed
within braces `\{' and `\}'.

	\begin{quote}
		\begin{alltt}
		block-name
		\{
		\}
		\end{alltt}
	\end{quote}

Case is significant throughout the layout file. Lines not enclosed within
a block that begin with `\texttt{\#}' are comment lines. All text
within a block is significant.

\texttt{format-p} will store all blocks for possible use; but the ones
listed in
the following section are the default blocks that \texttt{format-p}
will use when generating output.  No blocks are required, but if you want
something to appear you have to define a known block.

Value-registers may be included in all blocks except for the \emph{options}
block.  A value-register will be substituted during run-time with the most
current value read from the input stream. The value-register
specification is:

	\begin{quote}
		\texttt{.valueregistername[:[-]width.precision]}
	\end{quote}

The width and precision specifiers are optional. If the width is
negative, the result will be right justified. The precision specifier
is only effective for fields which have a `.' in them; and is intended
for numeric fields. If you use a
precision specifier for a pure text field, you may not end up with
what you want.

\texttt{format-p} defines only one value-register -- \textsf{pageno}. All other
value-registers are defined from the Input Stream.

\subsection{The options block}

	The \emph{options} block provides general setup information about the
	layout of the output page. It is the only layout-block that cannot
	contain value-registers.

	Each entry in the options-block is of the form:

	\begin{quote}
		\texttt{optionkeyword = value}
	\end{quote}

	The following option keywords are interpreted:

		\begin{description}

			\item[pitch]
				This directs \texttt{format-p} to use the specified pitch.
				Currently only 10 or 12 pitch is supported, for backward
				compatibility with \texttt{pformat}. If this is
				undefined, no printer sequences
				will be generated to alter the current pitch settings
				of the printer.

			\item[pagelength]
				The length of a page.

			\item[pageheader-start]
				The line on the page of which the \emph{pageheader} block
				will start printing on.

			\item[body-start]
				The line on the page which the \emph{body} block will
				begin printing on.

			\item[body-end]
				The last line on the page of which the \emph{body}
				block will print on.

			\item[pagetrailer-start]
				The line on the page of which the \emph{pagetrailer} block
				will start printing on.

		\end{description}

	None of the numbers are required (if \textbf{pagelength} is zero,
	a free form output will result); but if you stuff up the line
	numbering, it's your own fault.

\subsection{The report-header block}

	The \emph{report-header} block is only printed once per run. If
	defined, it will be the first block printed.

\subsection{The report-trailer block}

	The \emph{report-trailer} block is only printed once per run. If
	defined, it will be the last block printed.

\subsection{The page-header block}

	The \emph{page-header} block is printed once per page. If defined,
	it will be the first block printed on a page (bar the
	\emph{report-header} block). If \textbf{pageheader-start} has been
	defined, the block will begin printing on that line.

\subsection{The page-trailer block}

	The \emph{page-trailer} block is printed once per page. If defined,
	it will be printed after the body section on a page.
	If \textbf{pagetrailer-start} has been
	defined, the block will begin printing on that line.

\subsection{The page-trailer-last block}

	If defined, the \emph{page-trailer-last} block will only be
	printed on the very last page of output. If not the
	\emph{page-trailer} block will be used instead.
	If \textbf{pagetrailer-start} has been
	defined, the block will begin printing on that line.

\subsection{The body-header block}

	If defined, the \emph{body-header} block will only be printed once
	per run on the body of the page, prior to any \emph{body} detail
	lines.

\subsection{The body-trailer block}

	If defined, the \emph{body-trailer} block will only be printed once
	per run in the body of the page; after all \emph{body} detail
	lines have been generated.

\subsection{The body block}

	The \emph{body} block is generally intended to print (repeated) detail
	lines. The block will only print within the area defined by
	\textbf{body-start} and \textbf{body-end} in the \emph{options}
	block.

%%%%%%%%%%%%%%%%%%%%%%%%%%%%%%%%%%%%%%%%%%%%%%%%%%%%%%%%%%%%%%%%%%%%%%
\section{Input stream}

\texttt{format-p} determines the output environment and run-time data by
reading in values from standard-input. The Input Stream may be a
file which may be handed to \texttt{format-p} with:

	\begin{quote}
		\texttt{format-p < inputfile}
	\end{quote}

Alternatively, application code may hand its output to
\texttt{format-p} by using \texttt{popen(3)}.

	\begin{quote}
		\begin{alltt}

		FILE *	fmt = popen ("format-p", "w");

		fprintf (fmt, "#options\(\backslash\)n");
		fprintf (fmt, "lpno=%d\(\backslash\)n", lpno);
		...

		fprintf (fmt, "#data\(\backslash\)n");
		fprintf (fmt, "%s=%s\(\backslash\)n", key, value);
		..
		\end{alltt}
	\end{quote}

The information read is divided
into 2 sections, \emph{options} and \emph{data}. Generally all lines
read in both sections from are of the form:

	\begin{quote}
		\texttt{keyword=value}
	\end{quote}

All whitespace is significant, so make sure that the keyword doesn't
contain any -- otherwise its value will never be printed.

\subsection{The options section}

	The options section determines the output environment for
	\texttt{format-p}.
	Generally, this means setting the output destination and possibly
	the output queue.
	\texttt{format-p} will start interpreting options when it
	sees:

		\begin{quote}
			\texttt{\#options}
		\end{quote}

	The following keywords are recognised:

		\begin{description}
			\item[output]
				Specifies the file to which the output is to go to.

				If the first character of the string is `\texttt{|}', then
				the text following will be treated as a filter program.
				Any executable
				capable of reading from standard input may be used.

				If undefined, the internal formatter will be used.

			\item[lpno]
				Mandatory if \textbf{output} is undefined. This directs
				\texttt{format-p} to use the specifed printer number.

			\item[file]
				Mandatory. This specifies the layout file to use.

		\end{description}

\subsection{The data section}

	Aside \textsf{pageno}, \texttt{format-p} holds no value-registers
	initially.
	These are defined as \texttt{format-p} reads in
	the data section from standard
	input. \texttt{format-p} will enter into data mode when it sees:

		\begin{quote}
			\texttt{\#data}
		\end{quote}

	Any information read from then on will update
	\texttt{format-p}'s value-registers.
	When \texttt{format-p} sees a blank line, it will then print the
	\emph{body} block once using the most current values of its
	value-registers. If an undefined value-register is used within
	the layout file, an empty string is substituted.

	If during data mode \texttt{format-p} encounters a keyword that
	begins with `\texttt{\%}', \texttt{format-p} will interpret this
	to be a \emph{directive}. There are 2 classes of directives. The
	directive

		\begin{quote}
			\texttt{\%reset}
		\end{quote}

	will reset \texttt{format-p}. All value-registers are cleared, 
	\textsf{pageno} is reset to 1, and all layout-bodies reset to
	default.

	The other class of directives instructs \texttt{format-p} to change
	the layout-block matching the
	keyword with the layout-block specified with the value. eg:

		\begin{quote}
			\texttt{\%body=cocl-body}
		\end{quote}

	will direct \texttt{format-p} to use the layout-block
	\emph{cocl-body} to replace the default \emph{body} layout-block.
	To revert back to using the default \emph{body} layout-block,
	either of the following directives may be issued:

		\begin{quote}
			\begin{alltt}
				%body=body
				%body=
				%body
			\end{alltt}
		\end{quote}

%%%%%%%%%%%%%%%%%%%%%%%%%%%%%%%%%%%%%%%%%%%%%%%%%%%%%%%%%%%%%%%%%%%%%%
\section{Application Interfaces}

	There are 2 application language interfaces provided to interact with
	\texttt{format-p}: C and C++. Any others will be greatly welcome.

\subsection{C Interface}

	The C interface requires the following header files in your
	application code.

		\begin{quote}
			\begin{alltt}
			#include    <stdio.h>
			#include    <FormatP.h>
			\end{alltt}
		\end{quote}

	All interaction with \texttt{format-p} should begin with a call to
	\texttt{FormatPOpen()} or \texttt{FormatPOpenLpNo()}. The former
	will allow you to direct \texttt{format-p} output to a file or a
	pipe; the latter will direct \texttt{format-p} output to a printer.
	These functions will initiate a session to a concurrent
	\texttt{format-p} process which you can submit information to.

	\texttt{FormatPClose()} will close your session once
	you're done with it.

	Data submissions to a \texttt{format-p} session may be done with
	the \texttt{FormatPSubmit()} functions. \texttt{FormatPSubmitTable()},
	in particular, will submit a table's current column-information to
	\texttt{format-p}. eg:

		\begin{quote}
			\begin{alltt}

			FILE *  formatp = FormatPOpen ("layout", "/dev/null", NULL);

			if (cc = find_hash (cumr, &cumr_rec, EQUAL, "r", hhcu_hash))
			    file_err (cumr, cc, PNAME);

			FormatPSubmitTable (formatp, cumr); /* submit column values */
			FormatPClose (formatp);

			\end{alltt}
		\end{quote}

	You may also change the default \emph{page-header},
	\emph{page-trailer} and \emph{body} blocks with
	\texttt{FormatPPageHeader()}, \texttt{FormatPPageTrailer()} and
	\texttt{FormatPBody()} functions respectively. If a NULL value is
	given as a alternative block, the default will be used.

\subsection{C++ Interface}

	The C++ interface requires the following header file in your
	application code.

		\begin{quote}
			\begin{alltt}
			#include    <FormatP.h>
			\end{alltt}
		\end{quote}

%%%%%%%%%%%%%%%%%%%%%%%%%%%%%%%%%%%%%%%%%%%%%%%%%%%%%%%%%%%%%%%%%%%%%%
\section{Example Layout File}

Here's an example of the layout file with all the blocks defined.

\begin{alltt}
#   $Id: format-p.tex,v 5.0 2001/06/19 08:22:47 robert Exp $
#
#   Aside from the option block, all blocks may have a entries
#   consisting of straight text and value-registers.
#
#   A value-register has the form:
#
#       .identifier[:format-spec]
#
#   The optional format-spec consists of [-]length.precision.
#   The optional `-' #  denotes left-justification.
#   eg:
#
#       .cumr_dbt_name
#       .cumr_est_no:-2
#       .cohr_no_kgs:-3.1
#
#   are all valid value-register specifications
#
#   A reserved value-register `.pageno' is defined by the layout
#   program, and will substitute for the current page.
#
options
\{
    pitch = 10
    pagelength = 66

    pageheader-start = 5
    body-start = 20
    body-end = 40
    pagetrailer-start = 50
\}

report-header
\{
\}

report-trailer
\{
\}

page-header
\{
                                   Page No: .pageno

         .cumr_dl_adr1
         .cumr_dl_adr2
         .cumr_dl_adr3
         .cumr_dl_adr4
\}

page-trailer
\{
       Continued
\}

page-trailer-last
\{
    Total: .eval_quote_nett_total
\}

body-header
\{
\}

body-trailer
\{
\}

body
\{
    .inmr_description:30  .coln_disc_pc:-6.2  .eval_line_nett_value:-10.2
\}

cocl-body
\{
    .cocl_comment
\}

\end{alltt}

%%%%%%%%%%%%%%%%%%%%%%%%%%%%%%%%%%%%%%%%%%%%%%%%%%%%%%%%%%%%%%%%%%%%%%
\section{Input Stream Example}

Here's an example of an Input Stream that \texttt{format-p}
expects to see. This example will use the internal formatter
to generate its output.

\begin{alltt}
#options
lpno=1
file=/tmp/layout
#data
cumr_dl_adr1=Logistic Software
cumr_dl_adr2=Suite 2205B West Tower
cumr_dl_adr3=PSE Centre
cumr_dl_adr4=Ortigas Centre
inmr_description=None
coln_disc_pc=10
eval_line_nett_value=200.1

inmr_description=Oranges
coln_disc_pc=0
eval_line_nett_value=150

%body=cocl-body
cocl_comment=High quality Californian oranges.

cocl_comment=Sweeter and juicier than you can imagine.

%body=
inmr_description=Pears
coln_disc_pc=20
eval_line_nett_value=10.50
eval_quote_nett_total=300
\end{alltt}

The same Input Stream may be altered slightly to generate a fax, with
only minor changes to the \emph{options} section. The \emph{data} section
left unchanged.

\begin{alltt}
#options
output=|sendfax -d 4159088
file=/tmp/layout
\end{alltt}

%%%%%%%%%%%%%%%%%%%%%%%%%%%%%%%%%%%%%%%%%%%%%%%%%%%%%%%%%%%%%%%%%%%%%%
\section{Further extensions}

\texttt{format-p} could possibly define
the character sequences to alter pitch sizes, enable
expanded printing, et al. This could be done by defining
additional value-registers for use within the layout file.

\end{document}
